\documentclass[10pt,letterpaper]{article}
\begin{document}
\title{USING THE DATABASE SYSTEM TO KEEP TRACK OF THE FARMERS RECORDS IN UGANDA.}
\author{by CSC006  }
\maketitle
\section{Introduction:}
Farming is one of the Ugandans most prominent sources of income. However, most farmers do not keep any records while those who are not familiar with the modern computing technology keep track of their records manually for example by taking notes in their books of account to make an accountability though this method has been used for many years, it is not efficient, is very time consuming and records are likely to get lost or destroyed. the solution to this problem would be an automated record keeping system which is in form of a database.
Good farmer is a website that allows any poultry farmer all over the country who has access to the internet to create an account. in so doing, a database is created and it allows a farmer to make entries, store vie and update records each time he/she visits the website. 
\section{Literature Review:}
i) Jaguza is a mobile app that was created by Ronald Katamba, after losing all his rabbitsfrom which he raised money that saw him through school. He thought a method to cub livestock theft but also to monitor the animals in birds of the farm.
the application is great and it goes as far as detecting illness in animals within a period of 48 hours.
Jaguza is a good app but along the way looks more towards detecting disease than compiling the farmars' records as started earlier in the objectives.    
\section{Facom(Farming Consult And Mat co):}
It was founded in 2006 and its major aim is to provide farming consultations and management services.
this company offers knowledge for individuals to start farming, professional advice and management as regarded to farming, and even goes out of its way to offer land for higher purchase to start commercial farming.
however, Facom fails to set up an interface for the farmers to enter and monitor their record.
\section{Objectives:}
1) establishing  asound record keeping.
2) to highlight the importance of record keeping in various aspects of poultry production.
3) to use as a management tool to keep their proper health records, undertake performance evaluation, as well as perform other important management functions required to run effective and efficient farm enterprise. 
\section{References}
a).Dakan, Edward Alton. Influence of high school vocational agriculture upon production and management practices. Unpublished M.S. thesis. Library, Iowa State University of Science and Technology, Ames, Iowa. 1956.
\\b).Christy, James Raymond. Competencies in farm business analysis needed by farmers. Unpublished M.S. thesis. Library, Iowa State University of Science and Technology, Ames, Iowa. 1966. 
\\c).Donhowe, Charles Edward. Certain economic principles in farm and home development. Unpublished M.S. thesis. 

\end{document}
 
